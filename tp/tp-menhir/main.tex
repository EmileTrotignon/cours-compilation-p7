\documentclass{article}
\usepackage{packages}
\usepackage{macros}
\usepackage{options}

\title{Compilation -- TP 1 ~: \\ OCamlLex et Menhir}

\author{Université Paris Diderot -- Master 1}

\date{(2020--2021)}

\begin{document}

\maketitle

Les objectifs de cette séance de travaux pratiques sont:
\begin{itemize}
\item[\faSignIn]
de comprendre les rôles respectifs de l'analyse lexicale et grammaticale
dans l'analyse syntaxique ;

\item[\faSignIn]
de prendre connaissance des formats d'entrée des outils OCamllex et Menhir ;

\item[\faSignIn]
de comprendre la sémantique des spécifications OCamllex ;

\item[\faSignIn]
de comprendre la sémantique des spécifications Menhir ;

\item[\faSignIn]
de résoudre un conflit d'analyse grammaticale en spécifiant
des priorités ;

\item[\faSignIn]
de résoudre un conflit d'analyse grammaticale en réécrivant
une grammaire.

\end{itemize}

\medskip

Les fichiers sources utilisés pour ces travaux pratiques sont sur le dépôt GIT à
l'emplacement \verb!doc/td-menhir/code!.

Pour réaliser ces travaux pratiques, votre environnement de travail doit
inclure:
%
\begin{itemize}
\item \verb!dune >= 1.6!
\item \verb!menhir >= 20181113!
\item \verb!ocaml >= 4.05.0!
\end{itemize}

Si vous travaillez depuis les machines de l'UFR, faites depuis la racine
de votre compte.
%
\begin{verbatim}
ln -s /ens/guatto/.opam
\end{verbatim}

La commande \verb!eval $(opam config env)! dans votre shell courant devrait
configurer correctement votre environnement de travail. Rajoutez cette ligne
à la fin de votre \verb!.bashrc!.

\begin{exercise}[Prise en main du code source Marthe]
  \-
  \begin{enumerate}
  \item
    %
    Quel est le rôle de chaque fichier source fourni?

  \item
    %
    Compilez le code source.

  \item
    %
    Expliquez les avertissements produits par la compilation.

  \item
    %
    Après avoir lu \verb!marthe.ml!, expliquez ce que produise les entrées
    suivantes:
    %
    \begin{itemize}
    \item \verb!37!
    \item \verb!1 + 2!
    \item \verb!1 +!
    \item \verb!1 + 2 + 3!
    \end{itemize}
  \end{enumerate}
\end{exercise}

\begin{exercise}[Compléter l'analyseur lexical]
  \-
  \begin{enumerate}
  \item
    Rajoutez une règle d'analyse lexicale pour reconnaître le
    caractère \verb!*!  comme le lexème \verb!STAR!. Même chose pour
    les parenthèses gauches et droites.

  \item
    %
    Rajoutez une règle d'analyse lexicale pour reconnaître le mot-clé
    \verb!sum!.  Comment s'assurer que la chaîne \verb!sum! est bien reconnu
    comme ce mot-clé et non comme un identificateur?
  \end{enumerate}
\end{exercise}

\begin{exercise}[Compléter l'analyseur grammatical]
  \-
  \begin{enumerate}
  \item
    Remplacez \verb!%left PLUS! par \verb!%right PLUS!. Quel est l'effet
    de ce changement?

  \item
    Introduisez une règle pour reconnaître une multiplication. En étudiant
    \verb!_build/target/parser.conflicts!, expliquez pourquoi cette introduction
    provoque un conflit ``shift/reduce''.

  \item
    Après avoir rappelé les règles de spécification de priorité en Menhir,
    résoudre le conflit introduit par la question précédente. Comment savoir
    si vous avez correctement résolu le conflit?

  \item
    Complétez la grammaire pour reconnaître les expressions parenthésées.

  \item
    Complétez la grammaire pour reconnaître les expressions de la
    forme \verb!sum (x, start, stop, body)!. Pourquoi cette nouvelle
    règle ne rentre-t-elle pas en conflit avec la règle introduite par
    la question précédente alors qu'elles partagent des lexèmes?
  \end{enumerate}
\end{exercise}

\begin{exercise}[Pour les plus rapides]
  \-
  \begin{enumerate}
  \item
    %
    Dans le premier cours, la syntaxe de Marthe était stratifiée à l'aide de
    plusieurs non terminaux (factor, term, expression).  Modifiez votre
    grammaire Menhir pour suivre cette grammaire stratifiée. Pourquoi cette
    stratification fait-elle disparaître les conflits? Est-ce une grammaire
    équivalente à la grammaire précédente?

  \item
    %
    Modifiez la fonction \verb!eval! pour qu'elle calcule l'entier correspondant
    à l'expression Marthe.

  \item
    %
    Comment testez l'expression de l'utilisateur avant de l'évaluer pour être
    sûr que son évaluation ne va pas échouée? Ecrivez une fonction \verb!check!
    qui réalise ce test et produit un message d'erreur explicatif si une erreur
    potentielle peut se produire. Comment se comporte votre fonction sur les
    entrées suivantes :
    \begin{itemize}
    \item \verb!x!
    \item \verb!sum (x, 0, 10, y)!
    \item \verb!sum (x, 0, -10, y)!
    \end{itemize}
    ?

  \end{enumerate}
\end{exercise}

\end{document}
